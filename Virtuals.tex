\documentclass[aps,prl,reprint,amsmath,amssymb]{revtex4-1}

\usepackage{epsfig,color,graphicx}
%\usepackage{algorithmic}
\usepackage{algorithm}
\usepackage{algpseudocode}
\usepackage{soul}
\setcitestyle{super}

% Mathematical symbols
\newcommand*{\imi}{i} % imaginary i
\newcommand*{\E}{\mathrm{e}}
% DIRAC NOTATION
% bra-ket vectors
\newcommand{\ket}[1]{\ensuremath{\vert #1 \rangle}}
\newcommand{\bra}[1]{\ensuremath{\langle #1 \vert}}
\newcommand{\braket}[2]{\ensuremath{\langle #1 \vert #2 \rangle}} % bra-ket inner product
\newcommand{\ketbra}[2]{\ensuremath{\vert #1 \rangle \langle #2 \vert}} % ket-bra outer product
% operators
\newcommand{\op}[1]{\ensuremath{\hat{#1}}} % operator

\makeatletter
\renewcommand\NAT@biblabelnum[1]{{(#1)}}
\makeatother

%text color
%\newcommand{\new}{\color{red}}
%\newcommand{\blue}{\color{blue}}
%\newcommand{\old}{\color{black}}

\begin{filecontents*}{NLMOs-ctrl.bib}
@Control{achemso-ctrl,
  ctrl-article-title = {yes}
}
\end{filecontents*}

\begin{document}
\nocite{achemso-ctrl}

\bibliographystyle{achemso}

\title{
[Efficient] variable-metric localization of occupied and virtual one-electron orbitals
}

\author{Ziling Luo}
\email{ziling.luo@mail.mcgill.ca}
\author{Rustam Z. Khaliullin}
\email{rustam.khaliullin@mcgill.ca}
\affiliation{Department of Chemistry, McGill University, 801 Sherbrooke St. West, Montreal, QC H3A 0B8, Canada}

\date{\today}

\begin{abstract}
Comparison of variable-metric localization algorithms for occupied and virtual orbitals based on Boys-Foster and Pipek-Mezey localization functions for gas-phase and periodic systems.
\end{abstract}


\maketitle

\section{Introduction}
 
Spatially localized orbitals are widely used in electronic structure theory.
Occupied localized orbitals help describe, visualize and classify chemical bonding between atoms thus facilitating our understanding of electronic-structure origins of observed properties of atomistic systems.~\cite{boys1960construction, edmiston1963localized, pipek1989fast, niessen1972density, weinhold2012natural}.
Occupied and virtual localized orbitals are crucial ingredients in all local electronic structure methods including Kohn-Sham density functional theory (DFT)~\cite{goedecker1994efficient, bowler2012methods, zalesny2011linear, pulay1986orbital, saebo2001low, pisani2005local, hampel1996local, forner1997numerical} and wavefunction-based electron correlation methods~\cite{saebo1993local, schutz1999low, hetzer2000low, schutz2001low}.
In these methods, it is the locality of one-electron orbitals that allows to dramatically reduce the computational cost of modeling of large systems.

%Spatially localized orbitals are known as localized molecular orbitals (LMOs) in the field of molecular quantum chemistry and maximally localized Wannier functions (MLWFs) in solid state physics and materials science~\cite{marzari2012maximally}.
%Here, they will be collectively referred to as LMOs whereas the eigenstates of the effective one-electron Hamiltonian will be called canonical molecular orbitals (CMOs) regardless of whether the system is isolated or treated with periodic boundary conditions.

The self-consistent field procedure at the heart of Kohn-Sham DFT and most single-reference electronic structure theories yields spatially delocalized one-electron states. Typically, these states are the eigenstates of the effective one-electron Hamiltonian and are known as canonical molecular orbitals (CMOs) in the case of molecules and Bloch orbitals in the case of periodic systems. Here, they will be collectively referred to as CMOs. 
Traditionally, spatially localized orbitals are constructed by finding the unitary transformation of CMOs that minimizes the spread of individual orbitals. 
The unitary transformation can be applied to the set of occupied orbitals, generating a localized description of chemical bonding, or to the set of virtual orbitals, producing local anti-bonding orbitals. 

Multiple functions have been proposed to measure orbital locality in molecular systems. The most known are Boys-Foster~\cite{boys1960construction}, Edmiston-Ruedenberg~\cite{bytautas2002electron, bytautas2003split, edmiston1963localized}, Pipek-Mezey~\cite{pipek1989fast}, and Von Niessen~\cite{niessen1972density} functions. 
The Boys-Foster localization~\cite{boys1960construction} is perhaps the most popular because of the simplicity of its physical interpretation, low computational complexity and ease of implementation. 
The Pipek-Mezey localization~\cite{pipek1989fast}, which maximizes atomic charges~\cite{mulliken1955electronic, lowdin1950non, lehtola2014pipek} of each orbital, is also widely used because it does not mix LMOs representing $\sigma$ and $\pi$ bonds and thus gives a clear picture of bonding patterns. 
In the last decade, these well-established localization functions have been modified to reduce orbital tails and produce more uniform localization across a set of orbitals~\cite{jansik2011local, hoyvik2012orbital, hoyvik2012trust, hoyvik2013pipek, lehtola2013unitary, lehtola2014pipek}, especially virtual LMOs~\cite{jansik2011local, hoyvik2012orbital}.

For periodic systems, maximally localized Wannier functions (MLWFs)~\cite{marzari2012maximally,marzari1997maximally} represent the solid-state equivalent of Boys-Foster orbitals. 
%RZK: marzari1997maximally: Marzari, N., and D. Vanderbilt, 1997, Phys. Rev. B 56, 12 847.
For large supercells of condensed phase periodic systems, where electronic structure can be described with the $\Gamma$-point sampling of the Brillouin zone, generalized Pipek–Mezey Wannier functions have also been proposed~\cite{jonsson2017theory}. 
In this work, Wannier functions and localized molecular orbitals will be referred to as LMOs regardless of whether the system is isolated or treated with periodic boundary conditions.
 
Since CMOs are orthogonal and a unitary transformation applied to them during a localization procedure preserves the orbital metric, LMOs obtained in this way are orthogonal by construction.
Because of the the imposed orthogonality condition, orthogonal LMOs (OLMOs) exhibit small non-zero values even far away from their localization centers. 
These orthogonalization tails reduce orbital locality making orbital-based local correlation methods less computationally efficient. 
They also complicate the interpretation of chemically relevant electronic-structure information and make its transferability from one system to another more difficult. 
%A recent study has suggested that Pipek-Mezey scheme may produce semilocal orbitals in the case of virtual orbitals~\cite{hoyvik2013pipek} due to the system and basis set dependence property of the localization function. [RZK: There are lots of issues with PM orbitals or with virtual orbitals. Do we really want to single out this problem? We should mention this specific problem only if it is relevant to the work that we do here. Is it related to the orthogonalization tails?] 

To mitigate the undesirable orthogonality effects, metric-preserving unitary transformation has been applied to nonorthogonal orbitals~\cite{hoyvik2017generalising} and, in a more dramatic procedure, the unitary localization transformations have been replaced by with more general variable-metric nonsingular transformations~\cite{anderson1968self, diner1968fully, magnasco1974localized, payne1977hartree, mehler1977self, feng2004An_efficient, cui2010efficient, luo2020direct}. 
%RZK: occupied NLMOs only?
In the latter procedure, the generalization lifts the orthogonality constraint imposed on LMOs in the localization procedure and increases the number of degrees of freedom available to LMOs. 
It has been found that occupied nonorthogonal LMOs~(NLMOs) obtained in a variable-metric procedure are indeed about $10-30\%$ more localized than OLMOs if measured by the value of the Boys-Foster function~\cite{feng2004An_efficient, liu2000nonorthogonal, luo2020direct}. 

In contrast to metric-preserving unitary transformations, the variable-metric localization must be formulated to avoid
linear dependencies of orbitals during the minimization of the localization function~\cite{liu2000nonorthogonal, feng2004An_efficient, cui2010efficient, peng2013effective}. 
One approach to overcome the linear dependence problem, is to fix the centers of NLMOs~\cite{feng2004An_efficient, cui2010efficient} at the positions guessed from the knowledge of bonding patterns in the system~\cite{cui2010efficient} or at the centers of OLMOs~\cite{feng2004An_efficient}. 
Another more recent approach is to augment the localization function with a term that measures the deviation from the orthogonality and penalizes the states that are too close to linear dependence~\cite{luo2020direct}. The latter reformulation of variable-metric localization allows not only to determine optimal positions of the NLMOs' centers in a unconstrained and straightforward minimization procedure but to choose the desired balance between the orthogonality and locality of the orbitals. 

%RZK: Is this problem relevant to our current discussion? Unfortunately, The recent study has surprisingly shown that a pair of $\sigma$ and $\pi$ bonds tend to generated mixed $\tau$ and $\tau'$ orbitals for nonorthogonal LMOs~(NLMOs). NLMO-abbreviation

\textbf{Algorithms.} 

A variety of methods and algorithms have been developed to construct LMOs. 
In the case of OLMOs, the orthogonality is preserved through Jacobi sweeps~\cite{edmiston1963localized, barr1975improved} or by exponential parameterization of unitary transformations~\cite{berghold2000general}. 

Pioneering work by Luenberger [28] and Gabay [29] convert the constrained optimization problem [over unitary matrices] into an unconstrained problem, on an appropriate differentiable manifold.

%[RZZK: combine with previous?] For periodic systems, several efficient optimization algorithms have been proposed to construct Wannier function~\cite{marzari1997maximally, berghold2000general, marzari2012maximally, jonsson2017theory}. 
%
%RZK: What algorithm is used in:
%marzari2012maximally, marzari1997maximally
%jonsson2017theory
%
Jacobi sweeps -- an early localization algorithm cpnsisting of a consecutive mixing of orbitals pairs -- approach to localize orbitals. Because of its simplicity, the Jacobi optimization process is often slow and rarely converges in the case of virtual orbitals~\cite{RZK-citation-is-needed,subotnik?}. 

%RZK: Crazy rotations -- citation needed. What is the essence of the algorithm?

%RZK: What is the main advantage of 

Line methods 
The line search methods have been found to be more efficient than the two-by-two orbital transformation for occupied orbitals, even for the simplest steepest descent approach~\cite{edmiston1965localized}.
Conjugate gradient methods, which are typically outperform the steepest descent algorithm, fail to significantly improve the efficiency of the localization procedure~\cite{ryback1978application}.
Multiple second-order methods have also been suggested to accelerate the localization process of occupied orbitals by using the information from the gradient and Hessian.~\cite{leonard1982quadratically, kari1984parametrization}
Even though the Hessian involved methods are more time consuming compared with first-order methods, they successfully reduce the necessary amount of iterations by 1-2 orders of magnitude~\citet{leonard1982quadratically}.
The quasi-Newton methods such as Broyden-Fletcher-Goldfarb-Shanno~(BFGS) algorithm have also been implemented to localize molecular orbitals.~\cite{kari1984parametrization}

In the case of virtual OLMOs, 
Due to the fact that there are multiple not well separated local minimum of the localization functions in the case of virtual orbitals~\cite{subotnik2005fast} convergence is slow for line search methods even with the .
Moreover, the negative Hessian eigenvalues in the initial iterations of orbital localization prevent correctly predicting new directions, therefore the line search methods may fail in the early stage of the optimization process~\cite{RZK-citation-is-needed}.

More recently, the trust region methods have been demonstrated as an ideal optimization algorithm to localize both occupied and virtual orbitals with various localization functions.~\cite{jansik2011local, hoyvik2012trust, hoyvik2012orbital, hoyvik2013pipek}
However, the comparison between different localization functions has suggested the virtual orbitals obtained by Boy-Foster and Pipek-Mezey schemes are less localized than more advanced and complicated localization schemes.~\cite{hoyvik2013localized} 

Multiple optimization methods have also been implemented for localizing nonorthogonal orbitals.
The trust region methods have successfully applied to metric-preserving nonorthogonal occupied and virtual orbitals.~\cite{hoyvik2017generalising}
The line search methods such as conjugated gradient~\cite{liu2000nonorthogonal}, the Newton method~\cite{feng2004An_efficient} and BFGS method~\cite{cui2010efficient} have been suggested to localize variable-metric nonorthogonal occupied orbitals.

In this work, we designed, tested and compared a variety of line search and trust region algorithms to perform direct unconstrained variable-metric localization of occupied and virtual orbitals. [Currently, there are no reports/methods for variable-metric localization of virtual orbitals (is it true?). Here, we do not only discuss optimization algorithms but also compare, for the first time, virtual NLMOs generated with Boys and PM localization functions.]

[Line search methods depend on the line step size? Discussion ]
[Do we expect more significant reduction in orbital locality for virtual NLMOs than for occupied NLMOs?]

\section{Methodology}


\section{Results and discussion}

\section{Conclusions}


\section{Acknowledgments} 


%\bibliographystyle{abbrv}
\bibliography{Virtuals,Virtuals-ctrl}

\end{document}

